


Identification des compétences techniques
Analyse de l’expérience 
Identification des compétences relationnelles.
Difficultés rencontrées.
Succès/réussites, points de satisfaction
Mise en relation des croyances a priori avec l’expérience vécue ; évolution des croyances.
Identification des caractères et des traits de personnalité qui ont une importance dans cette expérience particulière.
Avez-vous perçu une cohérence entre la politique RSE affichée et votre expérience au sein de l’entreprise. 

\section{Analyse}


\subsection{Compétences Techniques}

Les compétences techniques qui m'ont été demandées ont été beaucoup orientées vers l'informatique.
On m'a d'abord demandé de réaliser une documentation entre deux versions d'Office : 2010 et 2019, ainsi que d'en faire un prospectus vulgarisé destiné à l'ensemble des salariés.
Il m'a donc fallu des compétences informatiques afin de réaliser la documentation mais également une compétence de vulgarisation afin de rendre les différences compréhensibles sans s'y attarder.

Il m'a également été demandé de mettre à jour des postes de Windows 7 à Windows 10, ainsi que de réaliser une mise à jour du BIOS.
Même si des procédures m'ont été données, je connaissais auparant toutes ces manipulations et cela a pu m'économiser beaucoup de temps et donc avancer plus vite sur mes tâches.

\subsection{Analyse de l'Expérience}

Cette expérience a pour moi été un très bon premier pas dans le monde de l'entreprise, et depuis le bas de l'échelle essayer d'en comprendre son fonctionnement global.

En effet, cela m'a permit de découvrir cet aspect du monde du travail que je ne connaissais pas, et d'apprendre la nécessité de beaucoup de compétences différents au sein d'une même entreprise.


\subsection{Compétences Relationnelles}

Concernant les compétences relationnelles, il a fallu être patiente et à l'écoute des problèmes des utilisateurs lors de l'inventaire et de l'installation de CRYHOD. 
Il ne suffit pas de rentrer dans les bureaux et de relever les informations relatives aux postes ou d'effectuer une installation. J'ai donc dû m'organiser afin de passer au bon moments dans les bureaux en prenant en compte que certains salariés étaient en télétravail donc absents du bureau.
Lors de l'installation j'ai également dû prendre le temps d'expliquer ce que je faisais, et expliquer comment on se sert du logiciel après l'installation.

De plus, certains salariés avaient des remarques, demandes ou problèmes et m'en ont parlé lors de mes passages pour que je puisse le faire remonter.
Par exemple le prêt de matériel est géré par le Département où j'effectuais mon stage, j'ai donc dû formuler les demandes à mes supérieurs, récupérer le matériel et aider les salariés à les installer en plus de mon travail d'inventaire.

J'ai donc bien approfondi mes compétences d'organisation et de vulgarisation, et je me suis rendue compte que savoir faire quelque chose ne suffit pas, il faut savoir l'expliquer et expliquer la démarche.

J'ai travaillé à la fois en équipe, où j'ai dû communiquer de manière claire et rapide, mais j'ai également beaucoup travaillé seule, j'ai donc dû faire preuve d'autonomie assez rapidement.



\subsection{Difficultés Rencontrées}

Ce stage m'a permit de me rendre compte de la complexité de fonctionnement d'une entreprise avec une si grande diversité de métiers.

La principale difficulté que j'ai rencontré a été de m'adapter à l'emploi du temps de tous les salariés lors des installations et inventaires.
En effet, en conciliant tous les télétravails et chômages partiels, réussir à voir l'ensemble des collaborateurs m'a demandé une organisation très précise.

La seconde difficulté, lors de la rédaction des supports Office, a été de me mettre à la place de certains salariés que le changement perturbe, et donc de créer des documents adaptés à tous.


\subsection{Succès et Réussites : Satisfaction}

Dans l'ensemble je suis satisfaite du travail que j'ai réalisé durant mon stage. Ma supérieure, Madame CORDEAU, m'a également fait des retours positifs sur le travail que j'ai fourni durant ce mois.

Les documents et rapports sur les versions d'Office sont clairs et attractifs à mon sens et à celui de mes supérieurs.

Concernant CRYHOD et Wyse, j'ai pu effectuer des installations que mes supérieurs m'ont remontées comme "propres" malgré l'arrêt prématuré de ces procédures.

Je suis donc contente du travail que j'ai fourni durant ces 4 semaines.


\subsection{Différences entre les Attentes et la Réalité}


\subsection{Caractères et Traits de personnalité requis dans l'entreprise}

De l'expérience que j'ai eu de l'entreprise, je dirais qu'il faut tout d'abord être curieux. En effet, c'est une entreprise avec une très grande variété de types de postes selon les services, et il faut savoir travailler ensemble.
Dans mon cas à l'informatique, il faut savoir comment travaillent les salariés des autres services afin de créer des outils et des logiciels adaptés à leur besoin et ne pas créer quelque chose qui ne s'adapterait pas à leur mode de travail.
Il faut donc s'intéresser à comment fonctionne chaque service.\newline

De plus, je pense qu'il faut savoir être patient. Chaque service travaillant sur des aspects complètement différents de la gestion de l'Aéroport, lors de travaux inter-services, il faut savoir expliquer ce que l'on fait, comment on travaille et laisser le temps à l'interlocuteur d'assimiler des informations qui ne sont pas dans son domaine de compétences.\newline

Et en dernier, je parlerai d'insistance. Surtout en ce moment avec une reprise exponentielle du traffic aérien non prévue, beaucoup de projets ont dû être mis en attente à cause de dysfonctionnements ou d'imprévus liés à la remontée du traffic.
Par exemple, depuis mars 2020 l'aéroport n'utilise plus que le Hall A, et la nécessité d'ouvrir à nouveau le hall B n'avait pas été prévue, ce hall a dû être prêt à ouvrir en une semaine, alors que les procédures habituelles comptent 3 à 4 semaines pour la réouverture habituelle.
Il faut donc savoir rester concentrer, et insister sur certaines choses même si beaucoup de problèmes apparaissent sur le côté.

\subsection{Application de la politique RSE}

Lors de l'inventaire j'ai eu l'occasion de passer dans les bureaux et discuter avec tous les salariés SA ADBM, j'ai pu constater que le tri sélectif est effectué partout et que d'une manière générale, les salariés sont impliqués dans la cause environnementale.

En effet, chaque service de l'entreprise à sa part à jouer dans ce défi :

\begin{itemize}
    \item Le Département des Directions Commerciales (DDC) : Service Environnement, Relation Riverains, Dématérialisation de la communication
    \item Le Département des Opérations Techniques (DOT) : Dématérialisation des serveurs, Consommation des équipements, Bâtiments aux normes environnementales
    \item Le Département Exploitation (DEX) : Sensibilisation du public au tri
    \item Le Département Administratif et Financier (DAF) : Service Qualité, Dématérialisation des factures et bulletins de paye
\end{itemize}

Chaque salarié est également sensibilisé lors de réunions annuelles sur la politique générale de l'Aéroport, avec comme thèmes : l'Environnement, le Bien-Être des Salariés et le Futur de l'entreprise.

Chacun est conscient des impacts de l'entreprise sur l'environnement et les riverains, et à chaque projet mené une partie du cahier des charges est liée à ces aspects.


