\chapter{Analyse}


\section{Compétences techniques}

Les compétences techniques qui m'ont été nécessiares étaient très orientées vers l'informatique.
On m'a d'abord demandé de réaliser une documentation entre deux versions d'Office (2010 et 2019), ainsi que d'en faire une plaquette vulgarisée destinée à l'ensemble des salariés.
Il m'a donc fallu des compétences informatiques afin de réaliser la documentation mais également une compétence de vulgarisation afin de rendre les différences compréhensibles sans difficultés.

Il m'a également été confié la tâche de mise à jour des postes de Windows 7 à Windows 10, ainsi que la réalisation des mises à jour du BIOS.
Même si des procédures m'ont été données, je connaissais auparant toutes ces manipulations et cela m'a permis de gagner beaucoup de temps et donc d'avancer plus vite sur mes tâches.

\section{Analyse de l'expérience}

J'ai apprécié ces quelques semaines au sein de la SA ADBM car cette expérience a constitué un très bon premier pas dans le monde de l'entreprise, et m'a permis d'en comprendre son fonctionnement global.

En effet, j'ai pu découvrir un aspect du monde du travail que je ne connaissais pas, et de me rendre compte de la nécessité d'avoir beaucoup de compétences différentes au sein d'une même entreprise.

J'ai également pu renforcer mes compétences tout en découvrant l'organisation d'un secteur qui m'intéresse grandement : l'aéronautique. Découvrir le travail de ces personnes qui travaillent dans l'ombre d'un secteur important a été très intéressant et m'a montré l'envers du décor.

Je suis donc très heureuse de ce stage, et je souhaite continuer à approfondir mes connaissances au cours de ma formation afin de travailler dans ce secteur plus tard.


\section{Compétences relationnelles}

Concernant les compétences relationnelles, il a fallu que je sois patiente et à l'écoute des problèmes des utilisateurs lors de l'inventaire et de l'installation de CRYHOD. 
En effet, il ne suffisait pas de rentrer dans les bureaux et de relever les informations relatives aux postes ou d'effectuer une installation. Il fallait expliquer au personnel toutes les manipulations et leurs procédures, ainsi que l'utilisation des nouveaux logiciels.

J'ai également dû m'organiser afin de passer au bon moment dans les bureaux en prenant en compte le fait que certains salariés étaient en télétravail et donc absents du bureau.

Toujours dans une démarche d'aide, certains collaborateurs m'ont demandé de réaliser des installations de matériel (écrans, postes, périphériques) qu'ils ne se sentaient pas capable de faire eux-même. J'ai donc pris du temps pour les aider à réaliser ces installations.

Cela m'a permis de bien approfondir mes compétences d'organisation et de vulgarisation, et je me suis rendue compte que savoir faire quelque chose ne suffit pas, il faut savoir l'expliquer et expliquer la démarche.

J'ai à la fois travaillé en équipe, où j'ai dû communiquer de manière claire et rapide, mais j'ai également beaucoup travaillé seule, j'ai dû aussi faire preuve d'autonomie assez rapidement.

Pour finir, les tâches m’ayant été confiées étant assez variées, j’ai donc dû faire preuve de polyvalence afin de pouvoir effectuer un travail rigoureux dans tous les domaines.

\section{Difficultés rencontrées}

Ce stage m'a permis de me rendre compte de la complexité du fonctionnement d'une entreprise avec une si grande diversité de métiers.

La première difficulté que j'ai rencontrée a été de m'adapter à l'emploi du temps de tous les salariés lors des installations et inventaires.
En effet, concilier toutes les contraintes de télétravails et chômages partiels pour réussir à voir l'ensemble des collaborateurs m'a demandé une organisation très précise.

La seconde difficulté, lors de la rédaction des supports Office, a été de me mettre à la place de certains salariés qui ne connaissent pas du tout la nouvelle version d’Office et dont le changement peut faire perdre en productivité, j’ai donc dû créer des documents adaptés à tous.

La plus grande difficulté que j’ai rencontrée était le stress. J’ai toujours été quelqu’un de très stressée, et j’étais très anxieuse à l’approche du stage : peur de mal faire, de ne pas m’intégrer, de décevoir, etc\dots
J’ai donc dû m’affronter moi-même pour surmonter ma peur et oser parler à mes collègues. Dans l’ensemble je dirais que je m’en suis bien sortie et que j’ai réussi à être plus à l’aise grâce à la bienveillance des personnes avec qui j’ai travaillé.


\section{Succès et réussites : satisfaction}

Dans l'ensemble je suis satisfaite du travail que j'ai réalisé durant mon stage. Ma supérieure, Madame CORDEAU, m'a également fait des retours positifs sur le travail que j'ai fourni durant ce mois.

Les documents et rapports sur les versions d'Office me semblent clairs et attractifs, cet avis est partagé par mes supérieurs. Concernant CRYHOD et Wyse, j'ai pu effectuer des installations que mes supérieurs m'ont remontées comme "propres" malgré l'arrêt prématuré de ces procédures.

Je suis donc contente du travail que j'ai fourni durant ces 4 semaines.


\section{Différences entre les attentes et la réalité}

Concernant les tâches demandées, je m’attendais, avant mon stage, à réaliser des travaux uniquement "techniques" comme de l’installation ou des mises à jour de postes en grande quantité. Cependant, j’ai également dû faire une documentation et des inventaires. Ce sont des tâches auxquelles je ne m’attendais pas, mais qui m’ont également formée sur d’autres aspects, plus humains.

J’ai toujours su proposer mon aide et me rendre utile à mes collègues de tous les services lorsque le travail qui m’avait été donné était terminé.

A propos des relations entre collègues, j’avoue avoir été surprise par la bonne ambiance qui régnait dans le service. L’entraide et la bienveillance étaient les maîtres mots. Je m’attendais plutôt à une relation stricte de travail, mais l’ambiance était excellente malgré la charge de travail et le professionnalisme de l’ensemble des collaborateurs.


\section{Caractères et traits de personnalité requis dans l'entreprise}

De part mon expérience dans l'entreprise, je dirais qu'il faut tout d'abord être curieux. En effet, c'est une société avec une très grande variété de types de postes dans les services, et il faut savoir travailler ensemble.
Dans mon cas à l'informatique, il faut savoir comment travaillent les salariés des autres secteurs d'activité afin de créer des outils et des logiciels adaptés à leur besoin et ne pas créer quelque chose qui ne s'adapterait pas à leur mode de travail.
Il faut donc s'intéresser à comment fonctionne chaque service.\newline

De plus, je pense qu'il faut savoir être patient. Chaque département travaillant sur des aspects complètement différents de la gestion de l'aéroport, lors de travaux inter-services, il faut savoir expliquer ce que l'on fait, comment on travaille et laisser le temps à l'interlocuteur d'assimiler des informations qui ne sont pas dans son domaine de compétences.\newline

Et en dernier trait de personnalité requis, je parlerai de ténacité. Tout particulièrement en ce moment, du fait de la forte reprise du traffic aérien difficile à prévoir. Beaucoup de projets ont dû être mis en attente suite à des dysfonctionnements ou des imprévus liés à la remontée du traffic.

Par exemple, depuis mars 2020 l’aéroport n’utilisait plus que le Hall A, et la nécessité d’ouvrir à nouveau le Hall B n’avait pas été prévue aussi tôt. Suite à une direction du CODIR\footnote{Comité de Direction} ce hall a dû être prêt à ouvrir en seulement une semaine, alors que les procédures habituelles comptent 3 à 4 semaines pour une réouverture. Il faut donc savoir rester concentrer, et prioriser ses tâches même si beaucoup d'imprévus apparaissent.

\newpage

\section{Application de la politique RSE}

Lors de l'inventaire j'ai eu l'occasion de passer dans les bureaux et de discuter avec tous les salariés de la SA ADBM. J'ai pu constater que le tri sélectif est effectué partout et que d'une manière générale, les salariés sont impliqués dans la cause environnementale.\newline

En effet, chaque service de l'entreprise à sa part à jouer dans ce défi :

\begin{itemize}
    \item La Direction (DIR) : prise de décision concernant l'ensemble des départements, service environnement, relation riverains, service qualité
    \item Le Département du Développement Commercial (DDC) : étude du traffic, dématérialisation de la communication
    \item Le Département des Opérations Techniques (DOT) : dématérialisation des serveurs, consommation des équipements électroniques, construction de bâtiments aux normes environnementales
    \item Le Département Exploitation (DEX) : sensibilisation du public au tri sélectif des déchets, pilotage des sociétés de nettoyage
    \item Le Département Administratif et Financier (DAF-RH) : dématérialisation des factures et bulletins de paye\newline
\end{itemize}

Chaque salarié est également sensibilisé lors de réunions annuelles sur la politique générale de l'aéroport, avec comme thèmes : l'environnement, la Qualité de Vie au Travail (QVT) et le futur de l'entreprise.

Chacun est conscient des impacts de l'entreprise sur l'environnement et les riverains. A chaque projet mené, une partie du cahier des charges est liée à ces aspects.\newline

Dans la globalité, j’ai trouvé que la politique RSE menée par l’entreprise est très bien appliquée, et même renforcée à chaque projet.
