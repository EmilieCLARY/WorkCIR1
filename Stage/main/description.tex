\section{Description}

De l’activité, de la structure d’accueil, des 
personnes impliquées, etc.
De la mission ou des tâches réalisées
Présentation de la démarche Responsabilité    
Sociale de l’entreprise (politique RSE)


\subsubsection*{Structure d'accueil}

J'ai donc effectué mon stage au sein du "DOT" : Département Opérations Techniques.

\subsubsection*{Personnes impliquées}


Mon tuteur était Serge CLARY, Chef de Projet Informatique, et ma supérieure Nathalie CORDEAU, Chef du Service Organisation, Informatique, Systèmes Industriels au sein du département.\footnote{Tous les organigrammes disponibles en Annexe}

J'ai travaillé en proche collaboration  avec :\newline
\begin{itemize}
    \item Monsieur Gurvan QUENET : Responsable Sécurité des Systèmes d'Information.
    \item Monsieur Marc RIVAULT : Administrateur Systèmes, Réseaux et Bases de données.
    \item Monsieur Yannick VALERY : Administrateur Systèmes, Réseaux et Bases de données.
    \item Monsieur Kamal MAHAMOUD : Technicien Support Informatique.
\end{itemize}

J'ai également été reçue par :

\begin{itemize}
    \item Monsieur CABANNE Olivier : Attaché Relations Riverains et Environnement
    \item Madame    Fabienne : 
\end{itemize}


L'avantage d'une entreprise avec une grande variété de postes, est que j'ai pu constater à quel point l'informatique est essentiel dans tous les services, autant sur le logiciel que sur le matériel.

\subsubsection*{Activités}

Les salariés de l'aéroport étant toujours sur Office 2010, ma première mission a été d'analyser les différences entre Office 2010 et 2019 afin de savoir quel type de formation ou documentation pourrait accompagner la transition, et ensuite de la réaliser.
J'ai donc créé un document détaillé de tous les changements entre ces versions, puis ensuite un "flyer" les résumant de manière simplifiée.\newline

Ma seconde mission était de mettre à jour des ordinateurs "CREWS", les passer de Windows 7 à Windows 10 en réinstallant d'autres logiciels. En banque d'enregistrement, ce sont des ordinateurs qui permettent d'enregistrer les bagages en soute et d'imprimer leurs identifications à partir d'un scan de la carte d'embarquement du passager.
J'ai commencé à faire quelques manipulations sur les ordinateurs CREWS : Mise à jour du BIOS, Installation de Windows à partir d'un logiciel de gestion : Ivanti.
Cependant je n'ai jamais pu les installer, des problèmes techniques ont été repérés plus tard dans l'installation, et nous avons donc dû suspendre le projet.\newline

Durant le premier confinement, certains salariés se sont vus attribués un ordinateur portable afin de faire du télétravail. Or plusieurs vols ont été enregistrés chez ces personnes, et au délà de la perte financière, la perte du disque dur représentait une perte d'informations internes et donc un problème de sécurité.
Il fallait donc régler le problème afin de ne pas risquer une fuite de données, CRYHOD était la solution. CRYHOD est un logiciel de cryptage de données de disques durs.
Gurvan QUENET, le Responsable Cybersécurité de l'aéroport m'a donc donné la mission d'installer ce logiciel sur les ordinateurs portables des salariés afin de sécuriser les disques durs et protéger les données de l'aéroport.
J'ai pu l'effectuer sur quelques postes, mais Monsieur QUENET m'a demandé d'arrêter à cause d'un problème de compatibilité sur certains logiciels, la mise en place a donc été retardée.\newline

Enfin, après l'annonce du confinement national en mars 2020, des outils informatiques ont été prêtés : Postes, Ecrans, Périphériques.
Ma dernière mission était de passer dans tous les bureaux de l'aéroport et de recenser le matériel présent avant de comparer avec celui prêté dans la base de données afin de vérifier que tout le monde a bien ramené les outils prêtés.

