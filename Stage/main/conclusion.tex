\chapter{Conclusion}

Ce stage de 4 semaines au sein de la Société Anonyme Aéroport de Bordeaux-Mérignac m’a été très bénéfique sur tous les points.\newline

Il m'a permis de participer contrètement à une activité technique dans un secteur grandissant : l'aéronautique. J'ai pu comprendre l'enjeux important de chaque action au cours de mes missions. J'ai beaucoup évolué au niveau technique, j'ai renforcé mes compétences durant mes tâches très variées. Grâce à celles-ci, j'ai pu me rendre compte que les tâches administratives me conviennent moins et que je préfèrerais m'orienter vers les métiers de l'informatique. Mon souhait de continuer dans la cybersécurité a été renforcé après les discussions avec Monsieur Gurvan QUENET, que je remercie pour son temps.

De plus, j'ai pu renforcer mes compétences relationnelles et cela m'apportera beaucoup dans le futur. Je me rends compte désormais à quel point le travail d'équipe et l'adaptation sont très importants en entreprise.

Ce temps passé en stage m’a permis de faire un premier pas dans le monde du travail et d’en découvrir les complexités. J’ai également pu découvrir la face cachée d’une grande entreprise : les domaines de chaque service et la hiérarchie.

Ces semaines ont également répondu aux questions que je me posais concernant les mesures environnementales. La SA ADBM est dans une période de transition pour rendre le futur meilleur, et je suis heureuse d'avoir pu y assister. L'évolution de leurs méthodes, et l'adaptation de l'entreprise m'a fait me rendre compte de l'importance de leurs missions pour l'environnement et le développement local.\newline

Cette expérience a été très enrichissante pour moi, et je remercie toutes les personnes qui ont contribuées à rendre mon stage le plus intéressant possible.
