\chapter{Conclusion}

Ce stage de 4 semaines au sein de la Société Anonyme Aéroport de Bordeaux-Mérignac a été très bénéfique.\newline

Il m'a permis de participer concrètement à une activité technique dans un secteur en croissance : l'aéronautique. J'ai pu comprendre l'enjeu important de chaque action au cours de mes missions. J'ai beaucoup progressé au niveau technique et j'ai approfondi mes compétences durant les missions très variées que l'on m'a confiées. Grâce à celles-ci, j'ai pu me rendre compte que les tâches administratives me conviennent moins et que je préfèrerais m'orienter vers les métiers de l'informatique. Après discussion avec Monsieur Gurvan QUENET, que je remercie pour le temps qu'il m'a consacré, mon souhait de continuer dans la cybersécurité a été renforcé.

De plus, j'ai pu renforcer mes compétences relationnelles et cela m'apportera beaucoup dans le futur. Je me rends compte désormais à quel point le travail d'équipe et l'adaptation sont très importants en entreprise.

Ce temps passé en stage m’a permis de faire un premier pas dans le monde du travail et d’en découvrir les complexités. J’ai également pu découvrir la face cachée d’une grande entreprise : les domaines de chaque service et la hiérarchie.\newline

Ces semaines ont également répondu aux questions que je me posais concernant les mesures environnementales. La SA ADBM est dans une période de transition pour rendre le futur meilleur, et je suis heureuse d'avoir pu y assister. L'évolution de leurs méthodes, et l'adaptation de l'entreprise face aux nouveaux enjeux environnementaux m'ont fait prendre conscience de l'importance de leurs missions pour l'environnement et le développement local.\newline

Cette expérience a été très enrichissante pour moi, et je remercie toutes les personnes qui par le temps qu'elles m'ont consacrées ont contribuées à rendre mon stage le plus intéressant possible.