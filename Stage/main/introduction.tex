\chapter{Introduction}

\section{Objectifs donnés par l'ISEN}


Au terme de notre première année en Cycle Informatique et Réseaux à l’ISEN, il nous a été demandé d’effectuer un stage en entreprise d’une durée de 4 semaines consécutives.


Ce stage devait se faire dans le domaine de la manutention et le monde industriel au sein d’une entreprise employant 30 personnes au minimum. Suite à la difficulté de trouver un stage en période de pandémie, l’ISEN a élargi les recherches à des stages de travaux d’informatique basique.


Cela nous permet de vivre une expérience professionnelle au sein d’une entreprise, dans un poste d’exécutant ou de technicien afin de découvrir le monde de la production et de la fabrication.


Elle nous permet aussi de nous rendre compte de ce qu'est la vie en entreprise, des compétences techniques et relationnelles à acquérir pour nos futurs postes à responsabilités.


\section{Demande de stage}

Lorsque l’ISEN a élargi les critères de recherche de stage aux petits travaux d’informatique, je me suis tournée vers des entreprises qui étaient susceptibles d'avoir besoin d’aide sur de la remise à niveau de matériel ou des installations de nouveaux parcs informatiques.


J’ai finalement réalisé mon stage à l’aéroport de Bordeaux-Mérignac, et plus précisément au sein du Département des Opérations Techniques.


Pour l’obtention de ce stage, j’ai été mise en relation avec la responsable du Service Organisation, Informatique, Systèmes Industriels de ce département, Madame Nathalie CORDEAU, afin de réaliser un entretien à distance. Après l’entretien, elle m'a recontactée pour me donner son accord et discuter des modalités du stage.


\section{Objectifs personnels}

Mes objectifs personnels pour ce stage étaient axés autour de trois parties : découvrir le monde profesionnel, satisfaire mes supérieurs, et laisser une empreinte positive de mon passage.


Une première expérience dans le monde profesionnel avant d’arriver sur le marché du travail est un réel avantage sur le CV. En effet, l’aéroport étant un bassin d’emploi assez conséquent et varié, cela pourrait m’apporter un futur emploi ou stage au sein de cette entreprise.



\section{Plan du rapport}

Dans ce rapport de stage, je vais commencer par présenter la structure d'accueil : son histoire, ses infrastructures et quelques chiffres. Puis j'expliciterai plus précisément mon expérience en présentant les personnes impliquées et mes missions et tâches. J'évoquerai également la politique RSE de l'entreprise et je traiterai des riverains et de l'environnement.

Dans une seconde partie, je détaillerai mon analyse de ces 4 semaines de stage au cours de sous parties traitant des compétences techniques requises, de l'analyse de l'expérience, des compétences relationnelles demandées. Mais je parlerai aussi des difficultés rencontrées au cours de mon stage, de la satisfaction de mon travail et des différences entre mes attentes et la réalité. Pour finir je parlerai des caractères et traits de personnalité que je trouve importants dans l'entreprise ainsi que l'application de la démarche RSE par l'entreprise et ses salariés.

Pour clore ce rapport, je réaliserai une conclusion et mettrai les annexes à votre disposition.